\input{initconf}

\begin{document}

\maketitle % Muestra el Título
\newpage %inserta un salto de página
\tableofcontents % para generar el índice de contenidos
\listoffigures
\newpage
%*************************************************************

\section{Monitorización de sistemas Linux}
\subsection{Conociendo el subsistema de archivos}
\subsubsection{Cuestión 1}
5.a) ¿Qué archivo le permite ver qué programas se han
instalado con el gestor de paquetes? 5.b) ¿Qué significan las terminaciones .
1.gz o .2.gz de los archivos en ese directorio?
\subsubsection{Cuestión opcional 1}
Indique qué comandos ha utilizado para realizarlo así
como capturas de pantalla del proceso de reconstrucción del RAID.
\subsection{Progrmando tareas con CRON}
\subsubsection{Cuestión 2}
¿qué archivo ha de modificar para programar una tarea?
Escriba la línea necesaria para ejecutar una vez al día una copia del
directorio ~/codigo a ~/seguridad/\$fecha donde \$fecha es la fecha actual
(puede usar el comando date).
\subsection{Analizando qué ocurre en el kernel con DMESG}
\subsubsection{Cuestión 3}
Pruebe a ejecutar el comando, conectar un dispositivo USB y
vuelva a ejecutar el comando. Copie y pegue la salida del comando.
(considere usar dmesg | tail). Comente qué observa en la información
mostrada.
\section{Monitorizando Windows: PERFMON}
\subsection{Cuestión 4}
Ejecute el monitor de “System Performance” y muestre el
resultado. Incluya capturas de pantalla comentando la información que
aparece.
\subsection{Cuestión 5}
Cree un recopilador de datos definido por el usuario (modo
avanzado) que incluya tanto el contador de rendimiento como los datos de
seguimiento:
Todos los referentes al procesador, al proceso y al servicio web.
Intervalo de muestra 15 segundos
Almacene el resultado en el directorio Escritorio\\logs
Incluya las capturas de pantalla de cada paso.
\section{Monitorizando el hardware}
\subsection{Cuestión 6}
instale alguno de los monitores comentados arriba en su
máquina y pruebe a ejecutarlos (tenga en cuenta que si lo hace en la
máquina virtual, los resultados pueden no ser realistas). Alternativamente,
busque otros monitores para hardware comerciales o de código abierto para
Windows y Linux.
\section{Otros monitores del sistema}
\subsection{MUNIM}
\subsubsection{Cuestión 7}
Visite la web del proyecto y acceda a la demo que
proporcionan (http://demo.munin-monitoring.org/) donde se muestra cómo
monitorizan un servidor. Monitorice varios parámetros y haga capturas de
pantalla de lo que está mostrando comentando qué observa.
\subsection{NAGIOS}
\subsubsection{Cuestión opcional 2}
instale Nagios en su sistema (el que prefiera)
documentando el proceso y muestre el resultado de la monitorización de su
sistema comentando qué aparece.
\subsection{GANGLIA}
\subsubsection{Cuestión opcional 3}
Haga lo mismo que con Munin.
\subsection{ZABBIX}
\subsubsection{Cuestión opcional 4}
Prueba a instalar este monitor es alguno de sus tres
sistemas. Realice capturas de pantalla del proceso de instalación y comente
capturas de pantalla del programa en ejecución.
\subsection{CACTI}
\subsubsection{Cuestión opcional 5}
Pruebe a instalar este monitor es alguno de sus tres
sistemas. Realice capturas de pantalla del proceso de instalación y comente
capturas de pantalla del programa en ejecución.
\subsection{AWSTATS}
\subsubsection{Cuestión opcional 6}
Instale el monitor y muestre y comente algunas
capturas de pantalla.
\subsection{Monitorizando un servicio (o ejecución de un programa)}
\subsubsection{Cuestión 8}
Escriba un breve resumen sobre alguno de los artículos donde
se muestra el uso de strace o busque otro y coméntelo.
\section{Profiling}
\subsection{GPROF Y VALGRIND}
\subsubsection{Cuestión opcional 7}
Desarrolle una página en C o C++ y analice su
comportamiento
usando
valgrind.
Visite
http://www.cs.tut.fi/~jkorpela/forms/cgic.html para ver un ejemplo sencillo
de una página web generada por un programa escrito en C.
\subsection{PHP}
\subsubsection{Cuestión opcional 8}
Desarrolle un script en PHP y analice su ejecución
con alguno (o los dos) profilers.
\subsection{PYTHON}
\subsubsection{Cuestión opcional 9}
Escriba un script en
python
y
analice
su
comportamiento usando el profiler presentado.

\subsection{POWERSHELL}
\subsubsection{Cuestión opcional 10}
Escriba un script en PowerShell y analice su
comportamiento usando el profiler presentado.
\subsection{MySQL}
\subsubsection{Cuestión 9}
Acceda a la consola mysql (o a través de phpMyAdmin) y
muestre el resultado de mostrar el ”profile” de una consulta (la creación de
la BD y la consulta la puede hacer líbremente).
\subsection{MongoDB}
\subsubsection{Cuestión opcional 11}
Al igual que ha realizado el “profiling” con MySQL,
realice lo mismo con MongoDB y compare los resultados (use la misma
información y la misma consulta, hay traductores de consultas SQL a
Mongo).

%*************************************************************
\newpage
\bibliographystyle{ieeetr}
\bibliography{citas}

\end{document}
