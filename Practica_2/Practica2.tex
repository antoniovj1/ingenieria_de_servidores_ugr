\documentclass[paper=a4, fontsize=11pt]{scrartcl}
\usepackage[T1]{fontenc}
\usepackage[utf8]{inputenc}
\usepackage{lmodern}
\usepackage{multirow}
\usepackage[table,xcdraw]{xcolor}
\usepackage[spanish]{babel}
\usepackage{cite}
\usepackage{amsmath,amsfonts,amsthm} % Math packages
\usepackage{graphics,graphicx, float} %para incluir imágenes y colocarlas
\usepackage[backref,colorlinks=true,linkcolor=black,urlcolor=blue,citecolor=blue]{hyperref} %Para crear enlaces en el pdf
\usepackage[noabbrev,spanish]{cleveref}
\usepackage{url}
\usepackage[shortlabels]{enumitem}
\usepackage{appendix}
\usepackage{eurosym}
\usepackage{epsfig}
\usepackage{caption}
\usepackage{subcaption}

\renewcommand{\appendixname}{Anexo}
\renewcommand{\appendixtocname}{Anexo}
\renewcommand{\appendixpagename}{Anexo}

\numberwithin{figure}{section} % Number figures within sections (i.e. 1.1, 1.2, 2.1, 2.2 instead of 1, 2, 3, 4)
\numberwithin{table}{section} % Number tables within sections (i.e. 1.1, 1.2, 2.1, 2.2 instead of 1, 2, 3, 4)
\newcommand{\horrule}[1]{\rule{\linewidth}{#1}} % Create horizontal rule command with 1 argument of height

\title{
    \normalfont \normalsize
    \textsc{{\bf Ingeniería de Servidores (2015-2016)} \\ Grado en Ingeniería Informática \\ Universidad de Granada} \\ [25pt] % Your university, school and/or department name(s)
    \horrule{0.5pt} \\[0.4cm] % Thin top horizontal rule
    \huge Memoria Práctica 5 \\ % The assignment title
    \horrule{2pt} \\[0.5cm] % Thick bottom horizontal rule
}
\author{Antonio de la Vega Jiménez }

%*************************************************************


\begin{document}

\maketitle % Muestra el Título
\newpage %inserta un salto de página
\tableofcontents % para generar el índice de contenidos
\listoffigures
\listoftables
\newpage

%*************************************************************

\section{Instalación de servicios y configuraciones}
%_____________________________________________________________
\subsection{yum}
\subsubsection{Cuestión 1}
\textit{Liste los argumentos de yum necesarios para instalar, buscar y eliminar paquetes.}
\subsubsection{Cuestión 2}
\textit{Qué ha de hacer para que yum pueda tener acceso a Internet?(Pistas: archivo de configuración en /etc, proxy: stargate.ugr.es:3128). ¿Cómo añadimos un nuevo repositorio?}
%_____________________________________________________________
\subsection{apt}
\subsubsection{Cuestión 3}
\textit{Indique el comando para buscar un paquete en un repositorio y el correspondiente para instalarlo.}
\subsubsection{Cuestión 4}
\textit{Indiqué qué ha modificado para que apt pueda acceder a los servidores de paquetes a través del proxy. ¿Cómo añadimos un nuevo repositorio?}
%_____________________________________________________________
\subsection{Windows}
%_____________________________________________________________
\subsection{OpenSuse}
\subsubsection{Cuestión opcional 1}
\textit{¿Qué gestores utiliza OpenSuse?}


%*************************************************************
\section{Gestión de los cortafuegos (firewalls)}
\subsection{Cuestión 5}
\textit{¿Qué diferencia hay entre telnet y ssh?}
\subsection{Cuestión 6}
\textit{¿Para qué sirve la opción -X? Ejecute remotamente, es decir, desde la máquina anfitriona (si tiene Linux) o desde la otra máquina virtual, el comando gedit en una sesión abierta con ssh. ¿Qué ocurre?}
\subsection{Cuestión 7}
\texttt{Muestre la secuencia de comandos y las modificaciones a los archivos correspondientes para permitir acceder a la consola remota sin introducir la contraseña. (Pistas: ssh-keygen, ssh-copy-id).}
\subsection{Cuestión 8}
\texttt{¿Qué archivo es el que contiene la configuración de sshd? ¿Qué parámetro hay que modificar para evitar que el usuario root acceda? Cambie el puerto por defecto y compruebe que puede acceder.}
\subsection{Cuestión 9}
\textit{Indique si es necesario reiniciar el servicio ¿Cómo se reinicia un servicio en Ubuntu? ¿y en CentOS? Muestre la secuencia de comandos para hacerlo.}
\subsubsection{Utilidades: screen y terminator}
\paragraph{Cuestión opcional 2}
\textit{Instale y pruebe terminator. Con screen, pruebe su funcionamiento dejando sesiones ssh abiertas en el servidor y recuperándolas posteriomente.}
\subsubsection{Un poco de seguridad: fail2ban}
\paragraph{Cuestión opcional 3}
\textit{Instale el servicio y pruebe su funcionamiento.}

%*************************************************************
\section{Instalación del servicio de acceso remoto a la consola (Secure Shell)}

%*************************************************************
\section{Administración remota de windows}

%*************************************************************
\section{Instalación de un servidor Web básico}
%_____________________________________________________________
\subsection{Instalación de Apache + MySQL (o MariaDB) + PHP (o Python) en Linux (LAMP)}
%_____________________________________________________________
\subsection{Windows: IIS}
%_____________________________________________________________
\subsection{Windows y otros servidores web}
%_____________________________________________________________
\subsection{Java Servlets}
%_____________________________________________________________
\subsection{Otro tipo de Bases de datos}

%*************************************************************
\section{Manteniendo los servicios actualizados}

%*************************************************************
\section{Administración web}


%*************************************************************
\newpage
\bibliographystyle{ieeetr}
\bibliography{citas}

\end{document}
