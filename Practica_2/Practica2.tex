\input{initconf}

\begin{document}

\maketitle % Muestra el Título
\newpage %inserta un salto de página
\tableofcontents % para generar el índice de contenidos
\listoffigures
\listoftables
\newpage

%*************************************************************

\section{Instalación de servicios y configuraciones}

\subsection{yum}
%***********************************************
%    CUESTIÓN 1
%***********************************************
\subsubsection{Cuestión 1}
\textit{Liste los argumentos de yum necesarios para instalar, buscar y eliminar paquetes.}
\newline

Los argumentos necesarios son: \cite{manyum}
\begin{itemize}
  \item Instalar: \texttt{yum install nombrePaquete}
  \item Buscar: \texttt{yum search nombre}
  \item Eliminar: \texttt{yum remove nombrePaquete o yum erase nombrePaquete}
\end{itemize}


%***********************************************
%    CUESTIÓN 2
%***********************************************
\subsubsection{Cuestión 2}
\textit{Qué ha de hacer para que yum pueda tener acceso a Internet?(Pistas: archivo de configuración en /etc, proxy: stargate.ugr.es:3128). ¿Cómo añadimos un nuevo repositorio?}
\newline


\subsection{apt}
%***********************************************
%    CUESTIÓN 3
%***********************************************
\subsubsection{Cuestión 3}
\textit{Indique el comando para buscar un paquete en un repositorio y el correspondiente para instalarlo.}
\newline

Los comandos necesarios son: \cite{manapt1} \cite{manapt2}
\begin{itemize}
  \item Buscar: \texttt{apt-cache search expresiónRegular }
  \item Instalar: \texttt{apt-get install nombrePaquete}
\end{itemize}
%***********************************************
%    CUESTIÓN 4
%***********************************************
\subsubsection{Cuestión 4}
\textit{Indiqué qué ha modificado para que apt pueda acceder a los servidores de paquetes a través del proxy. ¿Cómo añadimos un nuevo repositorio?}
\newline


\subsection{Windows}

\subsection{OpenSuse}
%***********************************************
%    CUESTIÓN OP 1
%***********************************************
\subsubsection{Cuestión opcional 1}
\textit{¿Qué gestores utiliza OpenSuse?}
\newline
 
Los gestores de paquetes usados por OpenSuse son Zypper y Yast, aunque Yast realiza la misma tarea que Zypper pero con interfaz gráfica. \cite{os1} \cite{os2} \cite{os3}

\section{Gestión de los cortafuegos (firewalls)}
%***********************************************
%    CUESTIÓN 5
%***********************************************
\subsection{Cuestión 5}
\textit{¿Qué diferencia hay entre telnet y ssh?}
\newline

Tanto Telnet ( TELecommunication NETwork ) como SSH ( Secure Shell ) son protocolos de acceso remoto, la principal diferencia es la seguridad que ofrecen, Telnet no usa ningún tipo de cifrado en las comunicaciones, por lo que se pueden interceptar todos los datos de la comunicación ( incluyendo contraseñas ), debido a esto , su uso no es muy recomendable. SSH , al contrario que Telnet, si es un protocolo seguro, ya que todas las comunicaciones van cifradas.  \cite{sshtle}

%***********************************************
%    CUESTIÓN 6
%***********************************************
\subsection{Cuestión 6}
\textit{¿Para qué sirve la opción -X? Ejecute remotamente, es decir, desde la máquina anfitriona (si tiene Linux) o desde la otra máquina virtual, el comando gedit en una sesión abierta con ssh. ¿Qué ocurre?}
\newline

La opción -X de ssh sirve para que la aplicación se ejecute en el servidor remoto, pero la interfaz gráfica se visualice en el ordenador local ( X11 forwarding ). \cite{sshx} Al ejecutar gedit desde la terminal ssh, se muestra la aplicación como si se estuviese ejecutando en nuestro ordenador, pero en la ventana se indica que se esta ejecutando en otro ordenador ( on Ubuntu). 

%***********************************************
%    CUESTIÓN 7
%***********************************************
\subsection{Cuestión 7}
\textit{Muestre la secuencia de comandos y las modificaciones a los archivos correspondientes para permitir acceder a la consola remota sin introducir la contraseña. (Pistas: ssh-keygen, ssh-copy-id).}
\newline
%***********************************************
%    CUESTIÓN 8
%***********************************************
\subsection{Cuestión 8}
\textit{¿Qué archivo es el que contiene la configuración de sshd? ¿Qué parámetro hay que modificar para evitar que el usuario root acceda? Cambie el puerto por defecto y compruebe que puede acceder.}

%***********************************************
%    CUESTIÓN 9
%***********************************************
\subsection{Cuestión 9}
\textit{Indique si es necesario reiniciar el servicio ¿Cómo se reinicia un servicio en Ubuntu? ¿y en CentOS? Muestre la secuencia de comandos para hacerlo.}


\subsubsection{Utilidades: screen y terminator}
%***********************************************
%    CUESTIÓN OP 2
%***********************************************
\paragraph{Cuestión opcional 2}
\textit{Instale y pruebe terminator. Con screen, pruebe su funcionamiento dejando sesiones ssh abiertas en el servidor y recuperándolas posteriomente.}


\subsubsection{Un poco de seguridad: fail2ban}
%***********************************************
%    CUESTIÓN OP 3
%***********************************************
\paragraph{Cuestión opcional 3}
\textit{Instale el servicio y pruebe su funcionamiento.}


\section{Instalación del servicio de acceso remoto a la consola (Secure Shell)}


\section{Administración remota de windows}


\section{Instalación de un servidor Web básico}

\subsection{Instalación de Apache + MySQL (o MariaDB) + PHP (o Python) en Linux (LAMP)}
%***********************************************
%    CUESTIÓN 10
%***********************************************
\subsubsection{Cuestión 10}
\textit{Muestre los comandos que ha utilizado en Ubuntu Server y en CentOS (aunque en este último puede utilizar la GUI, en tal caso, realice capturas de pantalla)}

%***********************************************
%    CUESTIÓN 11
%***********************************************
\subsubsection{Cuestión 11}
\textit{Enumere otros servidores web y las páginas de sus proyectos (mínimo 3 sin considerar Apache, IIS ni nginx).}


\subsection{Windows: IIS}
%***********************************************
%    CUESTIÓN 12
%***********************************************
\subsubsection{Cuestión 12}
\textit{Compruebe que el servicio está funcionando accediendo a la MV a través de la anfitriona.}


\subsection{Windows y otros servidores web}

\subsection{Java Servlets}
%***********************************************
%    CUESTIÓN OP 4
%***********************************************
\subsubsection{Cuestión opcional 4}
\textit{Realice la instalación de uno de estos dos “web containers” y pruebe su ejecución.}


\subsection{Otro tipo de Bases de datos}
%***********************************************
%    CUESTIÓN OP 5
%***********************************************
\subsubsection{Cuestión opcional 5}
\textit{Realice la instalación de MongoDB en alguna de sus máquinas virtuales. Cree una colección de documentos y haga una consulta sobre ellos. ( http://docs.mongodb.org/manual/installation/ )}


\section{Manteniendo los servicios actualizados}
%***********************************************
%    CUESTIÓN 13
%***********************************************
\subsection{Cuestión 13}
\textit{Muestre un ejemplo de uso del comando (p.ej. http://fedoraproject.org/wiki/VMWare)}


\section{Administración web}
%***********************************************
%    CUESTIÓN 14
%***********************************************
\subsection{Cuestión 14}
\textit{Realice la instalación de esta aplicación y pruebe a modificar algún parámetro de algún servicio. Muestre las capturas de pantalla pertinentes así como el proceso de instalación.}

%***********************************************
%    CUESTIÓN 15
%***********************************************
\subsection{Cuestión 15}
\textit{Instale phpMyAdmin, indique cómo lo ha realizado y muestre algunas capturas de pantalla.Configure PHP para poder importar BDs mayores de 8MiB (límite por defecto). Indique cómo ha realizado el proceso y muestre capturas de pantalla.}


\subsection{Más administradores:ISPCONFIG, DIRECTADMIN, CPANEL, PARALLELS PLESK,... }
%***********************************************
%    CUESTIÓN 16
%***********************************************
\subsubsection{Cuestión 16}
\textit{Viste al menos una de las webs de los software mencionados y pruebe las demos que ofrecen realizando capturas de pantalla y comentando qué está realizando.}


\section{Automatización de tareas con scripts}
\subsection{Shells}
%***********************************************
%    CUESTIÓN 17
%***********************************************
\subsubsection{Cuestión 17}
\textit{Ejecute los ejemplos de find, grep y escriba el script que haga uso de sed para cambiar la configuración de ssh y reiniciar el servicio.}

%***********************************************
%    CUESTIÓN OP6
%***********************************************
\subsubsection{Cuestión opcional 6}
\textit{Muestre un ejemplo de uso para awk.}


\subsection{PHP}
\subsection{Python}
%***********************************************
%    CUESTIÓN 18
%***********************************************
\subsubsection{Cuestión 18}
\textit{Escriba el script para cambiar el acceso a ssh usando PHP o Python.}


\subsection{Windows PowerShell}
%***********************************************
%    CUESTIÓN 19
%***********************************************
\subsubsection{Cuestión 19}
\textit{Abra una consola de Powershell y pruebe a parar un programa en ejecución (p.ej), realice capturas de pantalla y comente lo que muestra.}

\subsection{Más automatización}


%*************************************************************
\newpage
\bibliographystyle{ieeetr}
\bibliography{citas}

\end{document}
