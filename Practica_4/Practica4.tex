\input{initconf}

\begin{document}

\maketitle % Muestra el Título
\newpage %inserta un salto de página
\tableofcontents % para generar el índice de contenidos
\listoffigures
\newpage
%*************************************************************

\section{Phoronix Suite}
\subsection{Cuestión 1}
\textit{Instale la aplicación. ¿Qué comando permite listar los benchmarks disponibles?}

\subsection{Cuestión opcional 1}
\textit{Seleccione, instale y ejecute uno, comente los resultados. Atención: no es lo mismo un benchmark que una suite, instale un benchmark.}

\section{Benchmarks y tests de estrés para webs}
\subsection{Apache benchmark}
\subsubsection{Cuestión 2}
\textit{De los parámetros que le podemos pasar al comando ¿Qué
significa -c 5 ? ¿y -n 100? Monitorice la ejecución de ab contra alguna máquina (cualquiera) ¿cuántos procesos o hebras crea ab en el cliente?}

\subsubsection{Cuestión 3}
\textit{Ejecute ab contra a las tres máquinas virtuales (desde el SO anfitrión a las máquina virtuales de la red local) una a una (arrancadas por separado) y muestre y comente las estadísticas. ¿Cuál es la que proporciona mejores resultados? Fíjese en el número de bytes transferidos, ¿es igual para cada máquina?}

\subsection{Gatling}
\subsubsection{Cuestión opcional 2}
\textit{¿Qué es Scala? Instale Gatling y pruebe los escenarios por defecto.}

\subsection{Jmeter}
\subsubsection{Cuestión opcional 3}
\textit{Lea el artículo y elabore un breve resumen.}
\subsubsection{Cuestión 4}
\textit{Instale y siga el tutorial en \url{http://jmeter.apache.org/usermanual/build-web-test-plan.html} realizando capturas de pantalla y comentándolas. En vez de usar la web de jmeter, haga el experimento usando alguna de sus máquinas virtuales (Puede hacer una página sencilla, usar las páginas de phpmyadmin, instalar un CMS, etc.).}

\section{Benchmarks para Windows}
\subsection{AIDA64 (Antiguo Everest)}
\subsubsection{Cuestión opcional 4}
\textit{Seleccione un benchmark entre SisoftSandra y Aida.
Ejecútelo y muestre capturas de pantalla comentando los resultados.}

\section{Más benchmarks}
\subsection{Cuestión 5}
\textit{Cuestión 5: Programe un benchmark usando el lenguaje que desee. El benchmark debe incluir:
\begin{enumerate}
  \item Objetivo del benchmark
  \item Métricas (unidades, variables, puntuaciones, etc.)
  \item Instrucciones para su uso
  \item Ejemplo de uso analizando los resultados
\end{enumerate}}
%*************************************************************
\newpage
\bibliographystyle{ieeetr}
\bibliography{citas}

\end{document}
