\documentclass[paper=a4, fontsize=11pt]{scrartcl}
\usepackage[T1]{fontenc}
\usepackage[utf8]{inputenc}
\usepackage{lmodern}
\usepackage{multirow}
\usepackage[table,xcdraw]{xcolor}
\usepackage[spanish]{babel}
\usepackage{cite}
\usepackage{amsmath,amsfonts,amsthm} % Math packages
\usepackage{graphics,graphicx, float} %para incluir imágenes y colocarlas
\usepackage[backref,colorlinks=true,linkcolor=black,urlcolor=blue,citecolor=blue]{hyperref} %Para crear enlaces en el pdf
\usepackage[noabbrev,spanish]{cleveref}
\usepackage{url}
\usepackage[shortlabels]{enumitem}
\usepackage{appendix}
\usepackage{eurosym}
\usepackage{epsfig}
\usepackage{caption}
\usepackage{subcaption}

\renewcommand{\appendixname}{Anexo}
\renewcommand{\appendixtocname}{Anexo}
\renewcommand{\appendixpagename}{Anexo}

\numberwithin{figure}{section} % Number figures within sections (i.e. 1.1, 1.2, 2.1, 2.2 instead of 1, 2, 3, 4)
\numberwithin{table}{section} % Number tables within sections (i.e. 1.1, 1.2, 2.1, 2.2 instead of 1, 2, 3, 4)
\newcommand{\horrule}[1]{\rule{\linewidth}{#1}} % Create horizontal rule command with 1 argument of height

\title{
    \normalfont \normalsize
    \textsc{{\bf Ingeniería de Servidores (2015-2016)} \\ Grado en Ingeniería Informática \\ Universidad de Granada} \\ [25pt] % Your university, school and/or department name(s)
    \horrule{0.5pt} \\[0.4cm] % Thin top horizontal rule
    \huge Memoria Práctica 5 \\ % The assignment title
    \horrule{2pt} \\[0.5cm] % Thick bottom horizontal rule
}
\author{Antonio de la Vega Jiménez }

%*************************************************************


\begin{document}

\maketitle % Muestra el Título
\newpage %inserta un salto de página
\tableofcontents % para generar el índice de contenidos
\listoffigures
\newpage
%*************************************************************

\section{Parámetros del sistema y su edición}
\subsection{Sistemas UNIX: Sysctl y /proc}
\subsubsection{Cuestión 1}
\textit{Al modificar los valores del kernel de este modo, no logramos que persistan después de reiniciar la máquina. ¿Qué archivo hay que editar para que los cambios sean permanentes?}
\subsubsection{Cuestión 2}
\textit{¿Con qué opción se muestran todos los parámetros modificables en tiempo de ejecución? Elija dos parámetros y expliqué, en dos líneas, qué función tienen.}
\subsection{Windows: Edición del registro}
\subsubsection{Cuestión 3}
\textit{Realice una copia de seguridad del registro y restaurela, ilustre el proceso con capturas.}
\subsubsection{Cuestión 4}
\textit{¿Cómo se abre una consola en Windows? ¿Qué comando hay que ejecutar para editar el registro? Muestre su ejecución con capturas de pantalla.}
\subsubsection{Cuestión 5}
\textit{Las cadenas de caracteres y valores numéricos tienen distintos tipos. Busque en la documentación de Microsoft y liste todos los tipos de valores.}
\section{Mejora de un servivio concreto}
\subsection{Servidor web: Apache e IIS}
\subsubsection{Cuestión 6}
\textit{Enumere qué elementos se pueden configurar en Apache y en IIS para que Moodle funcione mejor.}
\subsubsection{Cuestión 7}
\textit{Ajuste la compresión en el servidor y analice su
comportamiento usando varios valores para el tamaño a de archivo partir del cual comprimir. Para comprobar que está comprimiendo puede usar el navegador o comandos como curl (see url) o lynx. Muestre capturas de pantalla de todo el proceso.}
\subsection{Servicios de libre elección}
\subsubsection{Cuestión 8}
\textit{Usted parte de un SO con ciertos parámetros definidos en la instalación (Práctica 1), ya sabe instalar servicios (Práctica 2) y cómo monitorizarlos (Práctica 3) cuando los somete a cargas (Práctica 4). Al igual que ha visto cómo se puede mejorar un servidor web (Práctica 5 Sección 3.1), elija un servicio (el que usted quiera) y modifique un parámetro para mejorar su comportamiento. (9.b) Monitorice el servicio antes y después de la modificación del parámetro aplicando cargas al sistema (antes y después) mostrando los resultados de la monitorización.}
\subsubsection{Cuestión opcional 1}
\textit{Realice lo mismo que en la cuestión 8 pero para otro servicio.}
%*************************************************************
\newpage
\bibliographystyle{ieeetr}
\bibliography{citas}

\end{document}
